\documentclass[12pt,a4paper]{article}
\usepackage[utf8x]{inputenc}
\usepackage{ucs}
\usepackage[nohead,left=1.5cm,right=1.5cm,top=1.5cm,bottom=1.5cm]{geometry}
\usepackage{amsmath}
\usepackage{amsfonts}
\usepackage{amssymb}
\usepackage{polski}
\usepackage[polish]{babel}
\usepackage{color,graphicx}
\usepackage{subfigure}
\usepackage{hyperref}
\usepackage{indentfirst}
\bibliographystyle{plain}

\author{Mateusz Boryń, Mateusz Pruchniak}
\title{SPOP Projekt - Restauracja}

\begin{document}

\maketitle
\thispagestyle{empty}
\pagestyle{empty}

Aplikacja ma za zadanie wspomaganie obslugi rezerwacji miejsc w restauracji. Udostępnia podstawowe operacje jakimi są: dodawanie, usuwanie, modyfikacja oraz wyświetlanie rezerwacji i stolików. Istnieje również możliwość wyszukiwania rezerwacji po wisanym nazwisku lub wyszukiwania wolnych stolików w okreslonym przedziale czasu. Podczs działania programu wykorzystywane są dwie struktury \textit{Rezerwacja} i \textit{Stolik} oraz dwa pliki \textit{rezerwacje.txt}  i \textit{stoliki.txt}. \textit{Rezerwacja} - struktura opisująca pojedyńczą rezerwację oraz plik \textit{rezerwacje.txt} zawierają wszystkie utworzone rezerwacje. \textit{Stolik} - struktura opisująca każdy stolik w restauracji oraz powiązany jest z nią plik \textit{stoliki.txt} który zawiera wszystkie dodane stoliki. W połączeniu te struktury tworzą relacyjną bazę danych. Poszczególne pola struktur zostały opisane w kodzie programu. Program posiada zabezpieczenia przed wpisaniem błędnych danych na przykład takich jak id stolików, które nie istnieją, ale również posiada sprawdzanie poprawności wpisywanych dat oraz numerów telefonów do klientów. 
\\\\
{\bf Opis interfejsu:} \\
Program uruchamiamy poleceniem \textit{main}. Dostępne są następujące opcje:

\begin{enumerate}
\item Zarządzaie stolikami: dodawanie (akcja \texttt{dodajStolik}), wyświetlanie (\texttt{wyswietlStoliki}), edycja (\texttt{edytujStolik}), usuwanie stolików (\texttt{usunStolik})
\item Zarządzaie rezerwacjami: dodawanie (\texttt{dodajRezerwacje}), wyświetlanie (\texttt{wszystkieRezerwacje}), edycja (\texttt{modyfikujRezerwacje}), usuwanie rezerwacji (\texttt{usuwanieRezerwacji})
\item Wyszukiwanie: wyszukiwanie rezerwacji wg nazwiska (\texttt{wyszukajPoNazwisku}), wyszukiwanie wolnych stolików (\texttt{wyszukajPoDacie})
\end{enumerate} 


W każdym podmenu można wcisnąć \textit{0} aby powrócić do głownego menu, natomiast wciśniecie \textit{0} w głównym menu spowoduje wyjście z programu.
\\
\end{document}