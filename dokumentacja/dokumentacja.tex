\documentclass[12pt,a4paper]{article}
\usepackage[utf8x]{inputenc}
\usepackage{ucs}
\usepackage[nohead,left=1.5cm,right=1.5cm,top=1.5cm,bottom=1.5cm]{geometry}
\usepackage{amsmath}
\usepackage{amsfonts}
\usepackage{amssymb}
\usepackage{polski}
\usepackage[polish]{babel}
\usepackage{color,graphicx}
\usepackage{subfigure}
\usepackage{hyperref}
\usepackage{indentfirst}
\bibliographystyle{plain}

\author{Piotr Kałamucki, Filip Nabrdalik}
\title{SPOP Projekt - Planowanie zajęć}

\begin{document}

\maketitle
\thispagestyle{empty}
\pagestyle{empty}
\section{Opis projektu}
Celem projektu jest stworzenie aplikacji umożliwiającej układanie planu zajęć w języku Haskell. Proces tworzenia grafiku może być manualny lub automatyczny. Aplikacja udostępnia prosty tekstowy interfejs użytkownika. Baza danych aplikacji składa się z plików o nazwach odpowiadających strukturom danych.
\textbf{Założenia:} zakładamy, że każda grupa musi realizować wszystkie przedmioty z puli stworzonych przedmiotów. Uznaliśmy że żadna grupa nie może mieć więcej niż 6 zajęć jednego dnia. Zajęcia odbywają się od poniedziałku do piątku, każdego dnia dostępnych jest 12 godzin lekcyjnych.
\section{Implementacja}
Program podzielony jest na moduły: UI, grupy, zajęcia, przedmioty, sale i narzędzia. Każdy moduł zawiera funkcje oraz typy danych do obsługi wymaganej funkcjonalności, np: moduł grupa udostępnia funkcje: wczytajGrupy, dodajGrupe itp. Piliki z danymi zawierają przykładowe 
\section{Struktura projektu}
\end{document}